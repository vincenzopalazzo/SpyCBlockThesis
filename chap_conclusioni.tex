\chapter{Conclusioni}\label{chap:conclusioni}

Il lavoro condotto in questa tesi ha evidenziato in particolare la possibilità di analizzare la blockchain di Bitcoin in modo scalabile, inoltre questo studio ha approfondito lo l'analisis del linguaggio Bitcoin Script e sul modo in cui una transazione Bitcoin può essere spesa, facendo emergere la necessità di approfondire gli studi in questo campo.\\
I risultati ottenuti attraverso lo sviluppo del parser descritto in questo documento, sono andati ben oltre le nostre aspettative, perché dopo l'analisi dei numerosi esempi illustrati in questo documento, come BlockSci, non pensavamo di riuscire a ricavare un prototipo di un software scalabile e sopratutto che riuscisse ad essere eseguibile in macchine comuni eseguendo l'analisi con tempi ragionevoli.\\
Guardando alle possibili estensioni del lavoro svolto, si può pensare subito di evolvere il parser ad un'implementazione completa multiprocessore, ciò migliorerebbe notevolmente le prestazioni di deserializzazione.\\
Inoltre attraverso l'utilizzo del prototipo SpyCBlock sono emerse una serie di problematiche riguardante la costruzione del grafo degli address, questo lascia spazio ad un implementazione di un sottomodulo per la deserializzazione approfondita degli script utilizzati in Bitcoin. Questo implica anche lo sviluppo di un metodo per la memorizzazione delle transazione deserializzate dal parser; con l'apporto di queste due modifica si potrebbe lavorare solo con i file serializzati dal nodo Bitcoin-core senza l'utilizzo del framework RPC.
