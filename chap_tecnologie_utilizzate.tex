\chapter{Tecnologie Utilizzate}\label{chap:tecnologieUtilizzate}

\section{Bitcoin-Cryptography-Library} \label{sec:cryptographyBitcoinLib}

La libreria Bitcoin-Cryptography-Library ha permesso di importare la crittografia di base utilizzata in Bitcoin, la libreria è stata sviluppata in C++11 ed in Java dallo sviluppatore Nayuki, la libreria risiede su Github sotto licenza MIT ed è progettata principalmente per microcontrollori ad 8 bit (Arduino, Atmel megaAVR/ATmega, etc.) con il supporto anche per dispositivi con architetture x86, x86-64 e 32-bit ARM.

\section{RapidJSON} \label{sec:rapidjsonLib}

La libreria RapidJSON ha permesso la decodifica in formato serializzato delle informazioni lette dal parser consentendo di creare una versione JSON della blockchain; è scritta in C++ dal team di sviluppo Tencent, viene rilasciata su Github sotto licenza BSD.
Con l’utilizzo della libreria siamo riusciti ad introdurre all’interno del parser una deserializzazione quasi real time di ogni blocco.


\section{Bitcoin-api-cpp} \label{sec:bitcoinApiLib}

La libreria bitcoin-api-cpp implementa un wrapper per il framework RPC di Bitcoin-core, risiede su Github sotto licenza MIT e viene implementata in C++ dallo sviluppatore Krzysztof Okupski.
Con l'utilizzo di questa liberia siamo riusciti ad'interfacciarci con il nodo Bitcoin per deserializzare le informazioni relative allo script di blocco.

\section{SpyCBlockRPC} \label{sec:spycblockrpcLib}

Questa libreria rappresenta un sottomodulo della soluzione proposta, sviluppato per disaccoppiare l'implementazione del parser dal nodo Bitcoin-core.
La libreria risiede su Github sotto licenza Apache License 2.0, sviluppata in C++.

\section{JSON-RPC 1.0} \label{sec:jsonrpchttp}

Bitcoin-core offre un framwork RPC (Remote Procedure Call) basato sul protocollo HTTP per consentire di utilizzare servizi offerti dal nodo Bitcoin, tipo: esaminare lo stato della blockchain oppure effettuare la crezione di transazioni in formato esadecimale.
Abbiamo utilizzato questo servizio offerto da Bitcoin-core per l'estrazione dell'address dallo script di blocco; inoltre ci ha consentito di ricavare i dati della transazionie precedente contenuta all'interno della transazione di input attualmente esaminata senza l'implementazione di una cache, questo però deteriora la velocità della soluzione proposta.

\section{Zlib} \label{sec:zlib}

La libreria zlib è una famosa libreria di compressione open surce, sviluppata da Jean loup Gailly e Mark Adler in C; raggiungendo uno standard RFC nel maggio 1996.
Viene inclusa all'interno la version 1.1 del Java Development Kit.
La libreria viene utilizzata per ridurre lo spazio richiesto dalle informazioni prodotte riguardanti i grafi, essendo uno standard possiede numerose implementazioni in vari linguaggi questo semplifica anche la fase di caricamento delle informazioni prodotte utilizzando linguaggi diversi dal C/C++.

\section{Bitcoin core library} \label{sec:bitcoinCoreLib}

Sono state utilizzate alcune parti del codice sorgente di Bitcoin core per diminuire il quantitativo di test da produrre e sopratutto diminuire il numero di errori possibili durante il parsing dei dati.\\
Le librerie estratte sono:
\begin{itemize}
  \item La libreria \say{serialize.h} con le relative dipendenze con cui è stato possibile serializzare/deserializzare ogni singolo tipo di dato nel formato corretto, ad esempio: le informazioni vengono deserializzate in \emph{little andian} e solo alcuni tipi di dati tipo l'hash viene deserializzato in \emph{big endian}.
  \item La libreria \say{endian.h} con le relative dipendenze con cui è stato possibile gestire rappresentazione dei dati in formato little endian e big endian.
  \item La libreria \say{uint256.h} con le relative dipendenze con cui è stato implementato il tipo di dato per rappresentare gli hash.
\end{itemize}
Tutte le librerie utilizzate ed estratte da Bitcoin core sono rilasciate sotto licenza MIT.
